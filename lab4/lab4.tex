\documentclass[12pt]{article}
\usepackage{ctex} % 中文支持
\usepackage{titling} % 标题格式定制
\usepackage{lipsum} % 示例文本
\usepackage{titlesec} % 标题格式定制
\usepackage{graphicx}
\usepackage{amssymb}
\usepackage{float}
\usepackage{color}
\usepackage{geometry}
\usepackage{multirow}
\geometry{left=2.5cm,right=2.5cm,top=2.54cm,bottom=2.0cm}

% 设置标题格式
\titleformat{\section}{\centering\Large\bfseries}{\thesection}{1em}{}
\titleformat{\subsection}{\centering\large\bfseries}{\thesubsection}{1em}{}
\titleformat{\subsubsection}{\centering\normalsize\bfseries}{\thesubsubsection}{1em}{}

% 调整标题位置
\setlength{\droptitle}{-4cm}

\begin{document}

\title{\huge{\textbf{2024秋数值代数-实验报告\#4}}}
\author{姓名:\underline{李奕萱} \hspace{1cm}学号:\underline{PB22000161} }
\date{\today}

\maketitle

运行环境:\underline{win11,vscode,py3}

\section{实验内容与要求}
\bf 
{病态线性方程组的求解}


问题提出:理论上的分析表明,求解病态的线性方程组是有困难的。实际情况是否如此,具体计算过程中究竟会出现怎样的现象呢?

{\textcolor{blue}实验内容}:考虑线性方程组~$\textcolor{blue}{Hx = b}$, 其中~H~为n阶~{\color{blue}Hilbert 矩阵}, 即

$$\textcolor{blue}{ H=(h_{ij})_{n\times n}, \qquad  h_{ij}=\frac{1}{i+j-1},\quad i,j=1,2,\cdots,n  }$$

这是一个著名的病态问题。通过先给{\color{blue}定解}, 比如{\color{red}\bf 取~$x$~的各个分量为1}, 再计算出右端向量~$\color{blue}b$~的办法给出一个精确解已知的问题.

{\color{blue} 实验要求}:

(1) 分别编写~Jacobi~迭代法,~Gauss-Seidel~迭代法以及最速下降法(SD)的一般程序({\color{red}不得使用符号运算});

(2) 所有迭代的{\color{red}初始向量均取为0向量, 停止条件为~$\|x^{(k+1)}  - x^{(k)} \|_1 < \epsilon \doteq 5\times 10^{-4}$~或迭代步数超过50万(可视为迭代失败)};

(3) 用以上三种迭代去求解前述的病态方程组,
分别取阶数~{\color{red} n=10,50,100,200,800};

(4) 列表给出{\bf 各自}{\color{blue}数值解}的计算误差({\bf\color{red}1-范数下})以及迭代步数;
    报告数值实验过程中可能出现的计算问题;
%以及本次编程实验的经验和教训.
%再分别取矩阵阶数n=10,50,200,500,重复(2);仍然用上述的两种计算方法去求解,请分别{\color{blue}报告各自的数值结果}({\bf\color{red}数值解和1-范数下的计算误差})以及{\color{blue}计算过程中可能出现的问题};

%(4) 对LU 分解,请分别报告n=6和10时的LU 分解的分解结果,即{\color{blue}给出对应的三角矩阵L和U}.
%
(5) {\bf 分析并比较以上三种迭代方法, 你能得出什么结论或经验教训.}
\clearpage


\section{数值结果}

{\bf --数值解误差及迭代步数}

\begin{table}[htb]
\begin{center}
\begin{tabular}{|c|c|c|c|} % 2列,每列居中对齐
\hline
               n              &  迭代法 & \quad 迭代步数 $\quad \;$ &  绝对误差~\color{blue}$\|x^{(k)} -x \|_1$ \\ \hline
   \multirow{3}{*}{n=10}      & \color{blue} Jacobi  &500000  & nan \\ \hline
                              & \color{blue}  Gauss-Seidel & 614 & 0.43287027795357225\\ \hline
                              & \color{blue}  SD & 317 & 0.1728045622586729\\ \hline \hline
    \multirow{3}{*}{n=50}      & \color{blue} Jacobi  &500000  &nan  \\ \hline
                              & \color{blue}  Gauss-Seidel & 2369 & 1.0927517603077161\\ \hline
                              & \color{blue}  SD &  625 &0.626794251788232 \\ \hline \hline
     \multirow{3}{*}{n=100}      & \color{blue} Jacobi  &500000  &nan \\  \hline
                              & \color{blue}  Gauss-Seidel & 3824 &1.9001978360241425 \\  \hline
                              & \color{blue}  SD &1109  & 0.9452248433669382\\ \hline \hline       
    \multirow{3}{*}{n=200}      & \color{blue} Jacobi  & 500000 &  nan\\  \hline
                              & \color{blue}  Gauss-Seidel & 5992 & 3.2994818463879247\\  \hline
                              & \color{blue}  SD & 1915 & 1.5143794069153476\\ \hline \hline            
    \multirow{3}{*}{n=800}      & \color{blue} Jacobi  & 500000 & nan \\  \hline
                              & \color{blue}  Gauss-Seidel &15944  &8.908334089350323 \\  \hline
                              & \color{blue}  SD &  5481 & 3.978635541478246\\ \hline %\hline                                                                       

\end{tabular}
\end{center}
\caption{数值结果}
\end{table}


\section{算法分析}
%从表1中,不难观察到
\begin{itemize}
    \item Jacobi迭代收敛性差
    \item SD迭代收敛效果最好
    \item Gauss\_Seidel迭代较Jacobi迭代更好,但在病态矩阵中跑不过SD迭代
\end{itemize}


\section{实验小结}
%\section{算法分析及小结}

 {计算过程中可能出现的问题(包括这次实验中的体会,收获或经验教训)}:
\begin{itemize}
\item  {\color{blue}Jacobi迭代:}

收敛性问题:如果矩阵不是对角占优的,Jacobi 方法可能无法收敛。此时,解的精度可能无法满足要求,或者收敛速度过慢。


数值稳定性:在高维问题中,Jacobi 方法可能会受到数值稳定性的影响,导致迭代结果不准确。
\item {\color{blue}Gauss\_Seidel迭代:} 

收敛性问题:虽然 Gauss-Seidel 方法对某些矩阵收敛速度较快,但对于某些特定问题,尤其是条件不好的矩阵,依然可能收敛缓慢或者根本不收敛。

迭代不稳定:对于某些稀疏矩阵或非常不规则的系统,Gauss-Seidel 方法可能变得不稳定,需要更多的迭代次数才能收敛。

\item {\color{blue}SD迭代:}

步长选择问题:步长(或学习率)是最速下降法的关键。如果步长过大,可能导致解不稳定或发散;步长过小,可能导致迭代收敛非常缓慢。

收敛性问题:对于条件较差的矩阵,最速下降法的收敛速度可能非常慢,甚至无法达到所需精度。
\hfill %\break

{比较三种算法的各自优缺点}:
\item  {\color{blue}Jacobi:}
优点:
简单易实现。
适用于对角占优的矩阵(对角占优矩阵是指对角线上的元素比其他元素绝对值大)。

缺点:
收敛性差,特别是对于对角不占优的矩阵。收敛速度较慢。
对于某些矩阵,可能需要较多的迭代才能得到足够精确的解。
\item {\color{blue} Gauss\_Seidel迭代:}

优点:
收敛速度比 Jacobi 迭代法快,特别是对于对角占优的矩阵。
适用于更广泛的矩阵类型,通常能更早收敛。

缺点:
需要逐步更新每个变量,这意味着在并行计算中不如 Jacobi 迭代法高效。
同样,若矩阵条件不好(比如不是对角占优的矩阵),可能导致收敛慢或无法收敛。

\item {\color{blue}SD迭代:}

优点:
对于某些问题,最速下降法的收敛速度非常快,特别是当问题的解是线性系统的最小二乘解时。
可以应用于非对称矩阵或不规则问题中,适应性强。

缺点:
对于条件不良的矩阵,最速下降法可能会收敛得非常慢,甚至在某些情况下无法收敛。
需要选择合适的步长,步长的选择直接影响到算法的收敛速度。如果步长过大,可能会跳过最优解;步长过小,可能会导致收敛速度非常慢。

\end{itemize}
{\color{blue}小结:}Jacobi迭代适合用在问题规模较小或矩阵满足对角占优的情况下。

对于收敛速度要求较高且矩阵条件较好的问题,Gauss-Seidel 方法非常有效。

最速下降法适用于最小二乘问题和那些要求快速找到近似解的问题,尤其是高维问题中,适用范围较广。

在选择算法时,应该根据问题的矩阵性质(如对角占优)来选择合适的算法,算法不合适
可能导致无法收敛或求解错误。

在病态矩阵的求解中,最速下降法表现最好。

\end{document}


