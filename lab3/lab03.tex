\documentclass[12pt]{article}
\usepackage{ctex} % 中文支持
\usepackage{titling} % 标题格式定制
\usepackage{lipsum} % 示例文本
\usepackage{titlesec} % 标题格式定制
\usepackage{graphicx}
\usepackage{amssymb}
\usepackage{float}
\usepackage{color}
\usepackage{geometry}
\usepackage{tabularx}
\geometry{left=2.5cm,right=2.5cm,top=2.54cm,bottom=2.0cm}

% 设置标题格式
\titleformat{\section}{\centering\Large\bfseries}{\thesection}{1em}{}
\titleformat{\subsection}{\centering\large\bfseries}{\thesubsection}{1em}{}
\titleformat{\subsubsection}{\centering\normalsize\bfseries}{\thesubsubsection}{1em}{}

% 调整标题位置
\setlength{\droptitle}{-4cm}

\begin{document}

\title{\huge{\textbf{2024秋数值代数-实验报告\#3}}}
\author{姓名:\underline{李奕萱} \hspace{1cm}学号:\underline{PB22000161} }
\date{\today}

\maketitle

运行环境:\underline{win11,vscode,py3}

\section*{实验内容与要求}

\textcolor{blue}{病态线性方程组的求解}


问题提出:理论上的分析表明,求解病态的线性方程组是有困难的。实际情况是否如此,具体计算过程中究竟会出现怎样的现象呢?

\textcolor{blue}{实验内容}:考虑线性方程组$\textcolor{blue}{Hx = b}$,其中H 为n阶\textcolor{blue}{Hilbert 矩阵},即

$$\textcolor{blue}{ H=(h_{ij})_{n\times n}\qquad  h_{ij}=\frac{1}{i+j-1},i,j=1,2,\cdots,n  }$$

这是一个著名的病态问题。通过先给\textcolor{blue}{定解}(例如取x的各个分量为1),再计算出右端向量b的办法给出一个精确解已知的问题

\textcolor{red}{实验要求}:

(1) 分别编写Doolittle LU 分解法和Cholesky 分解法的一般程序(\textcolor{red}{不得使用符号运算});

(2) 先取阶数n=6,分别用 LU 分解法、 Cholesky 分解法去求解上述的病态方程组$\textcolor{blue}{Hx = b}$;分别\textcolor{blue}{报告它们的数值结果}(即\textcolor{red}{数值解})以及它们在\textcolor{blue}{1-范数下的计算误差}。

(3) 再分别取矩阵阶数n=10和19,重复(2);仍然用上述的两种计算方法去求解,请分别\textcolor{blue}{报告各自的数值结果}(即\textcolor{red}{数值解})以及\textcolor{blue}{计算过程中可能出现的问题};

(4) 对LU 分解,请分别报告n=6和10时的LU 分解的分解结果,即\textcolor{blue}{给出对应的三角矩阵L和U}.

(5) \textcolor{blue}{适当地分析并比较两种计算方法},你能得出什么结论或经验教训.
\clearpage


\section{计算结果}
\begin{itemize}
  \item n=6:
\begin{table} [htb]
\begin{center}
\begin{tabularx}{\textwidth}{|c|X|}% 2列,每列居中对齐
\hline
\color{blue}$x$的精确解 & \color{blue} 如:(1,1,1,1,1,1) \\
\hline
\color{blue}LU分解法的数值结果      &\color{blue}(1. ,1. ,1., 1., 1., 1.) \\
\hline
\color{blue}LU分解法在1-范数下的计算误差     &\color{blue}9.20683640437403e-10  \\
\hline
\color{blue}Cholesky分解法的数值结果      &\color{blue}(1., 1., 1., 1., 1., 1.) \\
\hline
\color{blue}Cholesky分解法在1-范数下的计算误差      &\color{blue}2.3233626134100405e-10  \\
\hline
\end{tabularx}
\end{center}
\caption{n=6}
\end{table}

\item n=10:
\begin{table} [H]
\begin{center}
  \begin{tabularx}{\textwidth}{|c|X|} % 2列,每列居中对齐
\hline
\color{blue}$x$的精确解 & \color{blue} 如:(1,1,1,1,1,1,1,1,1,1) \\
\hline
\color{blue}LU分解法的数值结果      &\color{blue}(1.         1.00000003 0.99999926 1.00000669 0.99996817 1.00008732
0.99985697 1.00013801 0.99992765 1.00001589) \\
\hline
\color{blue}LU分解法在1-范数下的计算误差     &\color{blue}0.0004958885204457975  \\
\hline
\color{blue}Cholesky分解法的数值结果      &\color{blue}(1.         0.99999998 1.00000042 0.99999599 1.00001997 0.99994333
1.00009538 0.99990591 1.00005024 0.99998879)  \\
\hline
\color{blue}Cholesky分解法在1-范数下的计算误差      &\color{blue}0.00033200483286155436  \\
\hline
\end{tabularx}
\end{center}
\caption{n=10}
\end{table}

\item n=19:
\begin{table} [H]
\begin{center}
  \begin{tabularx}{\textwidth}{|c|X|} % 2列,每列居中对齐
\hline
\color{blue}$x$的精确解 & \color{blue} 如:(1,1,1,1,1,1,1,1,1,1,1,1,1,1,1,1,1,1,1) \\
\hline
\color{blue}LU分解法的数值结果      &\color{blue} (1.00000001  0.99999748  1.00012839  0.99738737  1.02742007  0.83181672
1.64320434 -0.57756443  3.48150342 -1.45313159  2.54414427  0.17616262
1.47110881  0.96453137  1.19957548 -0.19082302  2.56961822  0.13691487
1.17800559) \\
\hline
\color{blue}LU分解法在1-范数下的计算误差     &\color{blue}193.91944513459424  \\
\hline
\color{blue}Cholesky分解法的数值结果      &\color{blue} (1.0000001   0.99998416  1.00060052  0.99029064  1.08272459  0.59141146
2.20175296 -0.98548086  2.24431885  2.31638633 -1.61863164  1.52550547
2.47849426  0.70162635 -0.10294406  1.24332358  1.77261918  0.44639229
1.11162575) \\
\hline
\color{blue}Cholesky分解法在1-范数下的计算误差      &\color{blue}13.954703244325177  \\
\hline
\end{tabularx}
\end{center}
\caption{n=19}
\end{table}

\item n=6:


\resizebox{\textwidth}{!}{$
L=
\left[ \begin{array}{cccccc}

   1.        & 0.        & 0.        & 0.        & 0.        &
  0.         \\
  0.5       & 1.        & 0.        & 0.        & 0.        &
  0.         \\
  0.33333333& 1.        & 1.        & 0.        & 0.        &
  0.         \\
  0.25      & 0.9       & 1.5       & 1.        & 0.        &
  0.         \\
  0.2       & 0.8       & 1.71428571& 2.        & 1.        &
  0.         \\
  0.16666667& 0.71428571& 1.78571429& 2.77777778& 2.5       &
  1.         \\
\end{array}\right]
$}
\resizebox{\textwidth}{!}{$
U=
\left[ \begin{array}{cccccc}
   1.00000000e+00& 5.00000000e-01& 3.33333333e-01& 2.50000000e-01&
  2.00000000e-01& 1.66666667e-01 \\
  0.00000000e+00& 8.33333333e-02& 8.33333333e-02& 7.50000000e-02&
  6.66666667e-02& 5.95238095e-02 \\
  0.00000000e+00& 0.00000000e+00& 5.55555556e-03& 8.33333333e-03&
  9.52380952e-03& 9.92063492e-03 \\
  0.00000000e+00& 0.00000000e+00& 0.00000000e+00& 3.57142857e-04&
  7.14285714e-04& 9.92063492e-04 \\
  0.00000000e+00& 0.00000000e+00& 0.00000000e+00& 0.00000000e+00&
  2.26757370e-05& 5.66893424e-05 \\
  0.00000000e+00& 0.00000000e+00& 0.00000000e+00& 0.00000000e+00&
  0.00000000e+00& 1.43154905e-06 \\
\end{array}\right]
$}

\item n=10:


\resizebox{\textwidth}{!}{$
L=
\left[ \begin{array}{cccccccccc}
    1.        &  0.        &  0.        &  0.        &  0.        &
  0.        &  0.        &  0.        &  0.        &  0.         \\
  0.5       &  1.        &  0.        &  0.        &  0.        &
  0.        &  0.        &  0.        &  0.        &  0.         \\
  0.33333333&  1.        &  1.        &  0.        &  0.        &
  0.        &  0.        &  0.        &  0.        &  0.         \\
  0.25      &  0.9       &  1.5       &  1.        &  0.        &
  0.        &  0.        &  0.        &  0.        &  0.         \\
  0.2       &  0.8       &  1.71428571&  2.        &  1.        &
  0.        &  0.        &  0.        &  0.        &  0.         \\
  0.16666667&  0.71428571&  1.78571429&  2.77777778&  2.5       &
  1.        &  0.        &  0.        &  0.        &  0.         \\
  0.14285714&  0.64285714&  1.78571429&  3.33333333&  4.09090909&
  3.        &  1.        &  0.        &  0.        &  0.         \\
  0.125     &  0.58333333&  1.75      &  3.71212121&  5.56818182&
  5.65384615&  3.49999997&  1.        &  0.        &  0.         \\
  0.11111111&  0.53333333&  1.6969697 &  3.95959596&  6.85314685&
  8.61538461&  7.46666659&  3.99999893&  1.        &  0.         \\
  0.1       &  0.49090909&  1.63636364&  4.11188811&  7.93006993&
 11.63076923& 12.59999985&  9.52940824&  4.4999617 &  1.         \\
\end{array}\right]
$}
\resizebox{\textwidth}{!}{$
U=
\left[\begin{array}{cccccccccc}
   1.00000000e+00& 5.00000000e-01& 3.33333333e-01& 2.50000000e-01&
  2.00000000e-01& 1.66666667e-01& 1.42857143e-01& 1.25000000e-01&
  1.11111111e-01& 1.00000000e-01 \\
  0.00000000e+00& 8.33333333e-02& 8.33333333e-02& 7.50000000e-02&
  6.66666667e-02& 5.95238095e-02& 5.35714286e-02& 4.86111111e-02&
  4.44444444e-02& 4.09090909e-02 \\
  0.00000000e+00& 0.00000000e+00& 5.55555556e-03& 8.33333333e-03&
  9.52380952e-03& 9.92063492e-03& 9.92063492e-03& 9.72222222e-03&
  9.42760943e-03& 9.09090909e-03 \\
  0.00000000e+00& 0.00000000e+00& 0.00000000e+00& 3.57142857e-04&
  7.14285714e-04& 9.92063492e-04& 1.19047619e-03& 1.32575758e-03&
  1.41414141e-03& 1.46853147e-03 \\
  0.00000000e+00& 0.00000000e+00& 0.00000000e+00& 0.00000000e+00&
  2.26757370e-05& 5.66893424e-05& 9.27643785e-05& 1.26262626e-04&
  1.55400155e-04& 1.79820180e-04 \\
  0.00000000e+00& 0.00000000e+00& 0.00000000e+00& 0.00000000e+00&
  0.00000000e+00& 1.43154905e-06& 4.29464715e-06& 8.09375810e-06&
  1.23333457e-05& 1.66500167e-05 \\
  0.00000000e+00& 0.00000000e+00& 0.00000000e+00& 0.00000000e+00&
  0.00000000e+00& 0.00000000e+00& 9.00974946e-08& 3.15341229e-07&
  6.72727952e-07& 1.13522842e-06 \\
  0.00000000e+00& 0.00000000e+00& 0.00000000e+00& 0.00000000e+00&
  0.00000000e+00& 0.00000000e+00& 0.00000000e+00& 5.65997500e-09&
  2.26398940e-08& 5.39362123e-08 \\
  0.00000000e+00& 0.00000000e+00& 0.00000000e+00& 0.00000000e+00&
  0.00000000e+00& 0.00000000e+00& 0.00000000e+00& 0.00000000e+00&
  3.55145365e-10& 1.59814086e-09 \\
  0.00000000e+00& 0.00000000e+00& 0.00000000e+00& 0.00000000e+00&
  0.00000000e+00& 0.00000000e+00& 0.00000000e+00& 0.00000000e+00&
  0.00000000e+00& 2.22852500e-11 \\
\end{array}\right]
$}



\section{算法分析}

\begin{enumerate}
  \item LU分解和Cholesky分解时间空间复杂度相同,但LU分解要存储两个
  矩阵,所需存储空间会更大。
  \item n变大时,LU分解的误差会明显大于Cholesky分解。
  \item n=19时,H非正定,此时Cholesky分解会出问题,需要加一个小常数
  保持矩阵正定。
\end{enumerate}

\section{实验小结}

\textcolor{blue}{计算过程中可能出现的问题}:
\item 数值不稳定性:
Hilbert 矩阵是经典的病态矩阵,随着矩阵维度的增加,其条件数迅速变大,可能导致在计算过程中出现数值不稳定性。这会影响到 LU 和 Cholesky 分解的精度,特别是在浮点数计算中,可能出现舍入误差。
\item 非正定矩阵:
Cholesky 分解要求矩阵是对称正定矩阵。如果输入的矩阵不满足正定条件,比如 Hilbert 矩阵在某些情况下会趋近于奇异矩阵,Cholesky 分解将无法进行,触发求解错误。而 LU 分解没有这种限制,适用于更广泛的矩阵类型。


\hfill\break

\textcolor{blue}{分析比较两种算法}:
\item 适用范围:

LU 分解是一种适用于任意方阵的分解方法,可以分解不一定是对称正定的矩阵。因此它的适用范围更广。即使是奇异矩阵,LU 分解仍然可以处理。
Cholesky 分解 仅适用于对称正定矩阵,其使用场景更为有限。如果矩阵不是正定矩阵,Cholesky 分解会失败。

\item 计算效率:

Cholesky 分解 相比 LU 分解更高效,因为它仅需要对矩阵进行一次分解,计算复杂度是 
$𝑂(𝑛^3/3)$。
LU 分解 需要更多的计算量,复杂度为 
$𝑂(2𝑛^3/3)$,因为它需要分别计算上三角矩阵和下三角矩阵。
因此,对于对称正定矩阵,Cholesky 分解更快,而对于一般的矩阵,必须使用 LU 分解。

\item 数值稳定性:

LU 分解 在遇到病态矩阵(如 Hilbert 矩阵)时,可能会导致数值不稳定,尤其是在不进行部分选主元的情况下。为了提高稳定性,LU 分解通常结合选主元策略。
Cholesky 分解 的数值稳定性较高,前提是矩阵是正定的。当矩阵趋近于奇异时,Cholesky 分解容易失败。

\end{itemize}

\textcolor{blue}{总结}:
\begin{itemize}
\item LU 分解 是更通用的算法,适用于任意非奇异矩阵,但数值稳定性可能较差
,特别是在处理病态矩阵时。
\item Cholesky 分解 是一种更高效且更稳定的分解方法,但仅限于对称正定矩阵
。由于只需计算一个矩阵
,计算量相较于 LU 分解减少一半。

\end{itemize}

\end{document}
